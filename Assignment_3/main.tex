\documentclass[journal,12pt,twocolumn]{IEEEtran}

\usepackage{setspace}
\usepackage{gensymb}
\singlespacing
\usepackage[cmex10]{amsmath}

\usepackage{amsthm}

\usepackage{mathrsfs}
\usepackage{txfonts}
\usepackage{stfloats}
\usepackage{bm}
\usepackage{cite}
\usepackage{cases}
\usepackage{subfig}

\usepackage{longtable}
\usepackage{multirow}

\usepackage{enumitem}
\usepackage{mathtools}
\usepackage{steinmetz}
\usepackage{tikz}
\usepackage{circuitikz}
\usepackage{verbatim}
\usepackage{tfrupee}
\usepackage[breaklinks=true]{hyperref}
\usepackage{graphicx}
\usepackage{tkz-euclide}

\usetikzlibrary{calc,math}
\usepackage{listings}
    \usepackage{color}                                            %%
    \usepackage{array}                                            %%
    \usepackage{longtable}                                        %%
    \usepackage{calc}                                             %%
    \usepackage{multirow}                                         %%
    \usepackage{hhline}                                           %%
    \usepackage{ifthen}                                           %%
    \usepackage{lscape}     
\usepackage{multicol}
\usepackage{chngcntr}

\DeclareMathOperator*{\Res}{Res}

\renewcommand\thesection{\arabic{section}}
\renewcommand\thesubsection{\thesection.\arabic{subsection}}
\renewcommand\thesubsubsection{\thesubsection.\arabic{subsubsection}}

\renewcommand\thesectiondis{\arabic{section}}
\renewcommand\thesubsectiondis{\thesectiondis.\arabic{subsection}}
\renewcommand\thesubsubsectiondis{\thesubsectiondis.\arabic{subsubsection}}


\hyphenation{op-tical net-works semi-conduc-tor}
\def\inputGnumericTable{}                                 %%

\lstset{
%language=C,
frame=single, 
breaklines=true,
columns=fullflexible
}
\begin{document}

\newcommand{\BEQA}{\begin{eqnarray}}
\newcommand{\EEQA}{\end{eqnarray}}
\newcommand{\define}{\stackrel{\triangle}{=}}
\bibliographystyle{IEEEtran}
\raggedbottom
\setlength{\parindent}{0pt}
\providecommand{\mbf}{\mathbf}
\providecommand{\pr}[1]{\ensuremath{\Pr\left(#1\right)}}
\providecommand{\qfunc}[1]{\ensuremath{Q\left(#1\right)}}
\providecommand{\sbrak}[1]{\ensuremath{{}\left[#1\right]}}
\providecommand{\lsbrak}[1]{\ensuremath{{}\left[#1\right.}}
\providecommand{\rsbrak}[1]{\ensuremath{{}\left.#1\right]}}
\providecommand{\brak}[1]{\ensuremath{\left(#1\right)}}
\providecommand{\lbrak}[1]{\ensuremath{\left(#1\right.}}
\providecommand{\rbrak}[1]{\ensuremath{\left.#1\right)}}
\providecommand{\cbrak}[1]{\ensuremath{\left\{#1\right\}}}
\providecommand{\lcbrak}[1]{\ensuremath{\left\{#1\right.}}
\providecommand{\rcbrak}[1]{\ensuremath{\left.#1\right\}}}
\theoremstyle{remark}
\newtheorem{rem}{Remark}
\newcommand{\sgn}{\mathop{\mathrm{sgn}}}
\providecommand{\abs}[1]{\vert#1\vert}
\providecommand{\res}[1]{\Res\displaylimits_{#1}} 
\providecommand{\norm}[1]{\lVert#1\rVert}
%\providecommand{\norm}[1]{\lVert#1\rVert}
\providecommand{\mtx}[1]{\mathbf{#1}}
\providecommand{\mean}[1]{E[ #1 ]}
\providecommand{\fourier}{\overset{\mathcal{F}}{ \rightleftharpoons}}
%\providecommand{\hilbert}{\overset{\mathcal{H}}{ \rightleftharpoons}}
\providecommand{\system}{\overset{\mathcal{H}}{ \longleftrightarrow}}
	%\newcommand{\solution}[2]{\textbf{Solution:}{#1}}
\newcommand{\solution}{\noindent \textbf{Solution: }}
\newcommand{\cosec}{\,\text{cosec}\,}
\providecommand{\dec}[2]{\ensuremath{\overset{#1}{\underset{#2}{\gtrless}}}}
\newcommand{\myvec}[1]{\ensuremath{\begin{pmatrix}#1\end{pmatrix}}}
\newcommand{\mydet}[1]{\ensuremath{\begin{vmatrix}#1\end{vmatrix}}}
\numberwithin{equation}{subsection}
\makeatletter
\@addtoreset{figure}{problem}
\makeatother
\let\StandardTheFigure\thefigure
\let\vec\mathbf
\renewcommand{\thefigure}{\theproblem}
\def\putbox#1#2#3{\makebox[0in][l]{\makebox[#1][l]{}\raisebox{\baselineskip}[0in][0in]{\raisebox{#2}[0in][0in]{#3}}}}
     \def\rightbox#1{\makebox[0in][r]{#1}}
     \def\centbox#1{\makebox[0in]{#1}}
     \def\topbox#1{\raisebox{-\baselineskip}[0in][0in]{#1}}
     \def\midbox#1{\raisebox{-0.5\baselineskip}[0in][0in]{#1}}
\vspace{3cm}
\title{Assignment 3}
\author{Vijay Varma - AI20BTECH11012}
\maketitle
\newpage
\bigskip
\renewcommand{\thefigure}{\theenumi}
\renewcommand{\thetable}{\theenumi}
%
Download latex-tikz codes from 
%
\begin{lstlisting}
https://github.com/KBVijayVarma/EE3900/tree/main/Assignment_3
\end{lstlisting}
%
Download python codes from 
%
\begin{lstlisting}
https://github.com/KBVijayVarma/EE3900/tree/main/Assignment_3/code
\end{lstlisting}
\section*{\textbf{Problem (Construction Q 2.19)}}
Draw a circle of radius 3 and any two of its diameters. Draw the ends of these diameters. What figure do you get?
\section*{\textbf{Solution}}
Let us draw a Circle of radius 3 with centre O.

Let AB and CD be any two diameters of this circle such that AB and CD are not perpendicular. When we join the ends of these diameters, a Quadrilateral ACBD is formed.

Since the Diameters $\overline{\rm AB}$ and $\overline{\rm CD}$ of the circle are of equal length, hence the diagonals of the Quadrilateral ACBD are of equal length.

The intersection of the Diagonals is O, the Centre of the circle.

But all the radii of the Circle are of equal length, i.e., $\overline{\rm OA}$ = $\overline{\rm OB}$ = $\overline{\rm OC}$ = $\overline{\rm OD}$ = 3.

\begin{figure}[!h]
\centering
\includegraphics[width=\columnwidth]{fig1.jpg}
\caption{Rectangle}
\label{rectangle}
\end{figure}

So, the Diagonals of the Quadrilateral ACBD, $\overline{\rm AB}$ and $\overline{\rm CD}$ are of equal length and are bisecting each other.

We know that Rectangle is a Quadrilateral having Diagonals of equal length and the Diagonals should bisect each other.

Hence, the Quadrilateral ACBD is a \textbf{Rectangle}. 

This can be verified from above figure \ref{rectangle}. \\

Now, let us take two diameters PQ and RS of the circle such that PQ is perpendicular to RS, i.e., $\overline{\rm PQ} \perp \overline{\rm RS}$ .

When we join the ends of these diameters, a Quadrilateral PRQS is formed.

Since the Diameters $\overline{\rm PQ}$ and $\overline{\rm RS}$ of the circle are of equal length, hence the diagonals of the Quadrilateral PRQS are of equal length.

The intersection of the Diagonals is O, the Centre of the circle.

But all the radii of the Circle are of equal length, i.e., $\overline{\rm OP}$ = $\overline{\rm OQ}$ = $\overline{\rm OR}$ = $\overline{\rm OS}$ = 3.

So, the Diagonals of the Quadrilateral PRQS, $\overline{\rm PQ}$ and $\overline{\rm RS}$ are of equal length and are bisecting each other perpendicularly.

We know that Square is a Quadrilateral having Diagonals of equal length and the Diagonals should bisect each other perpendicularly.

Hence, the Quadrilateral PRQS is a \textbf{Square}.

\begin{figure}[!h]
\centering
\includegraphics[width=\columnwidth]{fig2.jpg}
\caption{Square}
\label{square}
\end{figure}

This can be verified from above figure \ref{square}.

\textbf{Therefore, the figure is a Rectangle if the Diameters are not perpendicular and is a Square if the Diameters are perpendicular.}


\end{document}