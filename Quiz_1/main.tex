\documentclass[journal,12pt,twocolumn]{IEEEtran}

\usepackage{setspace}
\usepackage{gensymb}
\singlespacing
\usepackage[cmex10]{amsmath}

\usepackage{amsthm}

\usepackage{mathrsfs}
\usepackage{txfonts}
\usepackage{stfloats}
\usepackage{bm}
\usepackage{cite}
\usepackage{cases}
\usepackage{subfig}

\usepackage{longtable}
\usepackage{multirow}

\usepackage{enumitem}
\usepackage{mathtools}
\usepackage{steinmetz}
\usepackage{tikz}
\usepackage{circuitikz}
\usepackage{verbatim}
\usepackage{tfrupee}
\usepackage[breaklinks=true]{hyperref}
\usepackage{graphicx}
\usepackage{tkz-euclide}

\usetikzlibrary{calc,math}
\usepackage{listings}
    \usepackage{color}                                            %%
    \usepackage{array}                                            %%
    \usepackage{longtable}                                        %%
    \usepackage{calc}                                             %%
    \usepackage{multirow}                                         %%
    \usepackage{hhline}                                           %%
    \usepackage{ifthen}                                           %%
    \usepackage{lscape}     
\usepackage{multicol}
\usepackage{chngcntr}

\DeclareMathOperator*{\Res}{Res}

\renewcommand\thesection{\arabic{section}}
\renewcommand\thesubsection{\thesection.\arabic{subsection}}
\renewcommand\thesubsubsection{\thesubsection.\arabic{subsubsection}}

\renewcommand\thesectiondis{\arabic{section}}
\renewcommand\thesubsectiondis{\thesectiondis.\arabic{subsection}}
\renewcommand\thesubsubsectiondis{\thesubsectiondis.\arabic{subsubsection}}


\hyphenation{op-tical net-works semi-conduc-tor}
\def\inputGnumericTable{}                                 %%

\lstset{
%language=C,
frame=single, 
breaklines=true,
columns=fullflexible
}
\begin{document}

\newcommand{\BEQA}{\begin{eqnarray}}
\newcommand{\EEQA}{\end{eqnarray}}
\newcommand{\define}{\stackrel{\triangle}{=}}
\bibliographystyle{IEEEtran}
\raggedbottom
\setlength{\parindent}{0pt}
\providecommand{\mbf}{\mathbf}
\providecommand{\pr}[1]{\ensuremath{\Pr\left(#1\right)}}
\providecommand{\qfunc}[1]{\ensuremath{Q\left(#1\right)}}
\providecommand{\sbrak}[1]{\ensuremath{{}\left[#1\right]}}
\providecommand{\lsbrak}[1]{\ensuremath{{}\left[#1\right.}}
\providecommand{\rsbrak}[1]{\ensuremath{{}\left.#1\right]}}
\providecommand{\brak}[1]{\ensuremath{\left(#1\right)}}
\providecommand{\lbrak}[1]{\ensuremath{\left(#1\right.}}
\providecommand{\rbrak}[1]{\ensuremath{\left.#1\right)}}
\providecommand{\cbrak}[1]{\ensuremath{\left\{#1\right\}}}
\providecommand{\lcbrak}[1]{\ensuremath{\left\{#1\right.}}
\providecommand{\rcbrak}[1]{\ensuremath{\left.#1\right\}}}
\theoremstyle{remark}
\newtheorem{rem}{Remark}
\newcommand{\sgn}{\mathop{\mathrm{sgn}}}
\providecommand{\abs}[1]{\vert#1\vert}
\providecommand{\res}[1]{\Res\displaylimits_{#1}} 
\providecommand{\norm}[1]{\lVert#1\rVert}
%\providecommand{\norm}[1]{\lVert#1\rVert}
\providecommand{\mtx}[1]{\mathbf{#1}}
\providecommand{\mean}[1]{E[ #1 ]}
\providecommand{\fourier}{\overset{\mathcal{F}}{ \rightleftharpoons}}
%\providecommand{\hilbert}{\overset{\mathcal{H}}{ \rightleftharpoons}}
\providecommand{\system}{\overset{\mathcal{H}}{ \longleftrightarrow}}
	%\newcommand{\solution}[2]{\textbf{Solution:}{#1}}
\newcommand{\solution}{\noindent \textbf{Solution: }}
\newcommand{\cosec}{\,\text{cosec}\,}
\providecommand{\dec}[2]{\ensuremath{\overset{#1}{\underset{#2}{\gtrless}}}}
\newcommand{\myvec}[1]{\ensuremath{\begin{pmatrix}#1\end{pmatrix}}}
\newcommand{\mydet}[1]{\ensuremath{\begin{vmatrix}#1\end{vmatrix}}}
\numberwithin{equation}{subsection}
\makeatletter
\@addtoreset{figure}{problem}
\makeatother
\let\StandardTheFigure\thefigure
\let\vec\mathbf
\renewcommand{\thefigure}{\theproblem}
\def\putbox#1#2#3{\makebox[0in][l]{\makebox[#1][l]{}\raisebox{\baselineskip}[0in][0in]{\raisebox{#2}[0in][0in]{#3}}}}
     \def\rightbox#1{\makebox[0in][r]{#1}}
     \def\centbox#1{\makebox[0in]{#1}}
     \def\topbox#1{\raisebox{-\baselineskip}[0in][0in]{#1}}
     \def\midbox#1{\raisebox{-0.5\baselineskip}[0in][0in]{#1}}
\vspace{3cm}
\title{Quiz - 1}
\author{Vijay Varma - AI20BTECH11012}
\maketitle
\newpage
\bigskip
\renewcommand{\thefigure}{\theenumi}
\renewcommand{\thetable}{\theenumi}
%
Download latex-tikz codes from 
%
\begin{lstlisting}
https://github.com/KBVijayVarma/EE3900/tree/main/Quiz_1
\end{lstlisting}
%
\section*{\textbf{Problem Q 2.30(D)}}
For each of the following systems, determine whether the system is (1) stable, (2) casual, (3) linear, (4) time invariant.

\textbf{(d)} $T(x[n]) = \sum_{k = n - 1}^{\infty} x[k]$
\section*{\textbf{Solution}}
Given $T(x[n]) = \sum_{k = n - 1}^{\infty} x[k]$
there

\textbf{(1) Stable}

\begin{align}
    y(n) &= \sum_{k = n - 1}^{\infty} x[k] \\
    y(n) &= x[k-1] + x[n] + x[n+1] + x[n+2] + \cdots + \infty
\end{align}

If x[n] is finite also, the above sum y(n) extends to infinity. \\

$\therefore$ The system $T(x[n]) = \sum_{k = n - 1}^{\infty} x[k]$ is \textbf{Unstable}. \\

\textbf{(2) Casual}

\begin{align}
    y(n) &= \sum_{k = n - 1}^{\infty} x[k] \\
    y(n) &= x[k-1] + x[n] + x[n+1] + x[n+2] + \cdots + \infty
\end{align}

Here the output y(n) depends on the future input (x[n+1], x[n+2], $\cdots$). \\

We know that if the output y(n) depends on future inputs then the system is a non casual system. \\

Hence, the given system $T(x[n]) = \sum_{k = n - 1}^{\infty} x[k]$ is a \textbf{Non - Casual} System. \\

\textbf{(3) Linear}

Let us take 
\begin{align*}
    y_1(n) = \sum_{k = n - 1}^{\infty} x_1[k] \\
    y_2(n) = \sum_{k = n - 1}^{\infty} x_2[k] \\
    x[n] = ax_1[n] + bx_2[n]
\end{align*}

Given $y(n) = \sum_{k = n - 1}^{\infty} x[k]$

\begin{align}
    y[n] &= \sum_{k = n - 1}^{\infty} [ax_1[k] + bx_2[k]] \\
    y[n] &= \sum_{k = n - 1}^{\infty} ax_1[k] + \sum_{k = n - 1}^{\infty} bx_2[k] \\
    y[n] &= ay_1[n] + by_2[n]
\end{align}

$\therefore y[n] = ay_1[n] + by_2[n]$

The given system holds the Superposition and Homogenity property.

Hence, the given system $T(x[n]) = \sum_{k = n - 1}^{\infty} x[k]$ is \textbf{Linear}. \\

\textbf{(4) Time Invariant}

Let 
\begin{align*}
    y_1(n) = \sum_{k = n - 1}^{\infty} x_1[k] \\ 
    y_2(n) = \sum_{k = n - 1}^{\infty} x_2[k] \\
    x_2[n] = x_1[n - t]
\end{align*}

where t is a real number

Now,

\begin{align}
    y_2(n) &= \sum_{k = n - 1}^{\infty} x_2[k] \\
    y_2(n) &= \sum_{k = n - 1}^{\infty} x_1[k - t]
\end{align}

Let us change the limits of the summation.

Let $k^{'} = k - t$

The lower limit $k = n - 1$ changes to $k^{'} = n - t - 1$.

The upper limit $k = \infty$ changes to $k^{'} = \infty$.

The above summation changes to 

\begin{align}
    y_2(n) &= \sum_{k^{'} = n - t - 1}^{\infty} x_1[k^{'}]
\end{align}

Now changing $k^{'}$ with k we get,

\begin{align}
    y_2(n) &= \sum_{k = n - t - 1}^{\infty} x_1[k] \\
    y_2(n) &= y_1(n-t)
\end{align}

Time Invariance can be verified from above.


Hence, the given system $T(x[n]) = \sum_{k = n - 1}^{\infty} x[k]$ is \textbf{Time Invariant}.

$\therefore$ The given system $T(x[n]) = \sum_{k = n - 1}^{\infty} x[k]$ is \textbf{Unstable, Non - Casual, Linear, Time Invariant}.

\end{document}